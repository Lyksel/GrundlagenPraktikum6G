
\section{Task 1: Inbetriebnahme von Keysight Advanced Design System (ADS)}
\subsection{Aufgabe 1.1: Installation von ADS}
Die Software \ac{ADS} dient zur Simulation von Schaltungen verschiedener Komplexitätsgrade. 
In diesem Versuch wird die Software verwendet, um eine Hochfrequenzschaltung zu simulieren und zu analysieren. 
Die Software bietet eine Vielzahl von Funktionen, darunter die Möglichkeit, Schaltungen zu entwerfen, 
S-Parameter zu simulieren und verschiedene Analysewerkzeuge zu verwenden.

\subsection{Aufgabe 1.2: Erstellen eines neuen Projekts}
Die Software ist auf den Rechnern im Labor bereits installiert gewesen. Nach dem Start der Software wird ein neues Projekt aus den
bereits zur Verfügung stehenden Workspaces erstellt. Diese sind auf der ILIAS-Seite des Praktikums in dem Dateiarchiv \texttt{TransmitterAmpDesign\_2024.zip} hinterlegt.
Die Datei wird entpackt und in der Software geöffnet. Außerdem werden die benötigten Bibliotheken aus dem Dateiarchiv \texttt{Infineon-RFTransistor-Keysight\_\ac{ADS}\_Design\_Kit-SM-v02\_10-EN.zip} geladen, 
diese stehen ebenfalls auf der ILIAS-Seite zur Verfügung.
\subsection{Aufgabe 1.3: Vertrautmachen mit der Software}
Schließlich werden die Tutorials 1 und 2 von \ac{ADS} durchgearbeitet, um sich mit der Software vertraut zu machen. 
Am Anfang der Schaltungsanalyse wird das Schema \texttt{TX\_Amp} benutzt.

\section{Task 2: Analyse des Datenblattes zu Transistor BFR181W}
Der maximal zulässige Kollektorstrom $I_{C,max}$ beträgt 20 mA. 
\section{Task 3: DC-Simulation und Wahl der Arbeitspunkte}
blabla
\section{Task 4: Simulation des S-Parameter}
blabla
\section{Task 5: Umsetzung der Schaltung auf dem PCB}
blabla
\clearpage