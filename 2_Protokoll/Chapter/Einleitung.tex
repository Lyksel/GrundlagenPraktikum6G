\section{Ziel des Versuchs}
Ziel des Versuchs ist es, einen HF-Verstärker zu simulieren und dessen Eigenschaften zu analysieren. Dabei wird ein BJT-Transistor verwendet, um die Verstärkung des Signals zu erhöhen.
Der Versuch umfasst die Analyse der Eigenschaften des vorliegenden Transistors der Reihe BFR181W, die Berechnung der Widerstände zur Anpassung des Arbeitspunkts,
um eine optimale Verstärkung zu erzielen, sowie die Durchführung einer S-Parameter-Simulation zur Überprüfung der Verstärkung und Stabilität des Verstärkers.

\section{Relevanz und Anwendungsbereiche von HF-Verstärkern}
Ein HF-Verstärker ist ein elektronisches Gerät, das Hochfrequenzsignale verstärkt. Diese Signale liegen typischerweise im Frequenzbereich von 3~kHz bis 300~GHz und finden Anwendung in verschiedenen Bereichen, wie der Kommunikationstechnik, Radartechnologie, Satellitenkommunikation sowie in der Medizintechnik.
HF-Verstärker sind entscheidend für die Signalübertragung und -verarbeitung in modernen Kommunikationssystemen. Sie werden eingesetzt, um schwache Signale zu verstärken, die von Antennen empfangen werden, und um sicherzustellen, dass die Signale über große Entfernungen übertragen werden können.

In unserem Versuch ist es von großer Bedeutung, die Eigenschaften des BJT-Transistors zu analysieren und die Widerstände so zu dimensionieren, dass ein stabiler Arbeitspunkt erreicht wird, damit bei späteren Versuchen eine Übertragung einer Bilddatei bei einer Frequenz von 1{,}25~GHz möglich ist.
\clearpage
