Zusammenfassend lässt sich sagen, dass die Simulation des HF-Verstärkers erfolgreich war. Die Aufgaben wurden sinnvoll bearbeitet und stimmen mit der Theorie im Rahmen der Messungenauigkeit überein.

Zum ersten mal in in der Geschichte unseres langen Studium hatten wir die berauschende Erfahrung einen HF-Verstärker in der freien Wildbahn zu beobachten.
In den Vorlesungen und Prüfungen mussten wir schon oft die Arbeitspunkteinstellung durchrechnen. Doch durch die E12 Reihe wurde es deutlich schwieriger
die richtige Dimensionierung der Widerstände zu finden. Diese Herausforderung war uns zum Anfang des Versuches nicht bewusst.
Wir haben alle drei unseren Horizont erweitert und freuen uns auf mehr:)




\clearpage