
\section{Zusammenfassung der wichtigsten Erkenntnisse}
\section{Reflexion und mögliche Verbesserungen}
\section{Eigene Reflexion}
\subsection{Erik}
\subsection{Farhad}
Zusammenfassend zu der HF-Simulation lässt sich sagen, dass die Simulation der Hochfrequenzschaltung mit \ac{ADS} erfolgreich durchgeführt wurde. Die Dimensionierung der Widerstände wurde erfolgreich durchgeführt, um einen stabilen Arbeitspunkt zu erreichen. Auch die S-Parameter-Simulation zeigte ein positives Gain von 11,266 dB bei einer Frequenz von 1{,}25 GHz.
Die Ergebnisse der Simulation stimmen mit den theoretischen Berechnungen überein.
Die Software \ac{ADS} erwies sich als nützliches Werkzeug zur Analyse und Simulation von Hochfrequenzschaltungen.
\subsection{Lukas}
Der Versuch hat mir zum ersten Mal anschaulich gezeigt, wie ein Verstärker in der Praxis tatsächlich funktioniert. In zahlreichen Vorlesungen haben wir zwar bereits die Arbeitspunkteinstellung und die Dimensionierung von Widerständen theoretisch durchgerechnet, jedoch fehlte bisher der praktische Bezug. Es war daher sehr interessant zu sehen, wie diese Konzepte im realen Aufbau umgesetzt werden und welche Herausforderungen dabei auftreten.

Besonders spannend fand ich die Aufgabe, die Widerstände so zu dimensionieren, dass ein stabiler Arbeitspunkt erreicht wird. Dabei habe ich festgestellt, dass dies in der Praxis deutlich anspruchsvoller ist als in der Theorie, da die Bauteile nicht beliebig gewählt werden können, sondern auf die Werte der E12-Reihe beschränkt sind. Dies erfordert zusätzliche Überlegungen und Anpassungen, um die gewünschten Schaltungseigenschaften zu realisieren.


\clearpage