
\section{Inbetriebnahme von Keysight Advanced Design System (ADS)}
\subsection{Installation von ADS}
Die Software \ac{ADS} dient zur Simulation von Schaltungen verschiedener Komplexitätsgrade. 
In diesem Versuch wird die Software verwendet, um eine Hochfrequenzschaltung zu simulieren und zu analysieren. 
Die Software bietet eine Vielzahl von Funktionen, darunter die Möglichkeit, Schaltungen zu entwerfen, S-Parameter zu simulieren und verschiedene Analysewerkzeuge zu verwenden.
\section{DC-Simulation}
\subsection{Erstellen eines neuen Projekts}
Die Software ist auf den Rechnern im Labor bereits installiert gewesen. 
Nach dem Start der Software wird ein neues Projekt aus den bereits zur Verfügung stehenden Workspaces erstellt. 
Diese sind auf der ILIAS-Seite des Praktikums in dem Dateiarchiv \texttt{TransmitterAmpDesign 2024.zip} hinterlegt. 
Die Datei wird entpackt und in der Software geöffnet. Außerdem werden die benötigten Bibliotheken aus dem Dateiarchiv \texttt{Infineon-RFTransistor-Keysight ADS Design Kit-SM-v02 10-EN.zip} geladen, diese stehen ebenfalls auf der ILIAS-Seite zur Verfügung.
\section{Analyse des Datenblattes zu Transistor BFR181W}
Die maximal zulässige Kollektorstrom $I_{C,\mathrm{max}}$ beträgt 20\,mA.
\section{S-Parameter Simulation}

blabla
\clearpage
