In diesem Kapitel wird die \ac{HF}-Simulation des Transistors BFR181W durchgeführt sowie die Dimensionierung der Widerstände $R_3$, $R_4$ und $R_5$ aus dem Kapitel 2: \enquote{Theoretische Grundlagen} überprüft und belegt.

\section{Inbetriebnahme von Keysight Advanced Design System (ADS)}
\subsection{Installation von ADS}
Die Software \ac{ADS} dient zur Simulation von Schaltungen unterschiedlichen Komplexitätsgrades. 
In diesem Versuch wird die Software verwendet, um eine Hochfrequenzschaltung zu simulieren und zu analysieren. 
Die Software bietet eine Vielzahl von Funktionen, darunter die Möglichkeit, Schaltungen zu entwerfen, S-Parameter zu simulieren und verschiedene Analysewerkzeuge zu nutzen.

\subsection{Erstellen eines neuen Projekts}
Die Software ist auf den Rechnern im ILH-Labor bereits installiert.\\
Nach dem Start der Software wird ein neues Projekt aus den bereits zur Verfügung stehenden Workspaces erstellt.\\
Diese sind auf der ILIAS-Seite des Praktikums im Dateiarchiv \texttt{TransmitterAmpDesign 2024.zip} hinterlegt.\\
Die Datei wird entpackt und in der Software geöffnet. Außerdem werden die benötigten Bibliotheken aus dem Dateiarchiv \texttt{Infineon-RFTransistor-Keysight ADS Design Kit-SM-v02 10-EN.zip} geladen. Diese stehen ebenfalls auf der ILIAS-Seite zur Verfügung.

\subsection{Vertrautmachen mit der Benutzeroberfläche}
Schließlich werden die Tutorials 1 und 2 von \ac{ADS} durchgearbeitet, um sich mit der Benutzeroberfläche und den grundlegenden Funktionen der Software vertraut zu machen.
Zu Beginn der Schaltungsanalyse wird das Schema \texttt{TX\_Amp\_Bias.dds} geöffnet. Dieses ist in Abbildung \ref{fig:TX_BIAS} zu sehen.

\begin{figure}[h]
    \centering
    \includegraphics[width=0.7\textwidth]{Pictures/TX_BIAS.png}
    \caption{Schaltbild der Schaltung \texttt{TX\_Amp\_Bias.dds} in \ac{ADS}.}
    \label{fig:TX_BIAS}
\end{figure}

\section{Analyse des Datenblattes zum Transistor BFR181W}
Um die Schaltung zu simulieren, wird der Transistor BFR181W aus der verfügbaren Bibliothek verwendet. Die genauen Parameter des Transistors werden aus dem Datenblatt entnommen.
Dieses steht ebenfalls auf der ILIAS-Seite des Praktikums zur Verfügung.

Die Tabelle \enquote{Maximum Ratings at $T_\mathrm{A}=25\,^\circ\mathrm{C}$, unless otherwise specified} unten links auf Seite 1 des Dokuments zeigt, dass der maximal zulässige Kollektorstrom $I_{C,\mathrm{max}}$ 20\,mA beträgt.

\section{DC-Simulation}
Im Folgenden wird eine DC-Simulation der später aufzubauenden Schaltung durchgeführt. 
Außerdem werden die Arbeitspunkte optimal durch die Anpassung der Widerstandswerte eingestellt.
Die DC-Simulation wird in \ac{ADS} durchgeführt, um die DC-Pegel der Schaltung zu überprüfen.

\subsection{Dimensionierung des Kollektorwiderstandes}
\label{sec:dimensionierung_kollektorwiderstand}
% Folgende Spannungswerte werden angenommen: FÜR LUKAS
% \begin{itemize}
%     \item $U_{CC} = 4.8\,\mathrm{V}$
%     \item $U_{BE} = 0.77\,\mathrm{V}$
% \end{itemize}
%
% Zuerst wird der Kollektorwiderstand $R_5$ so gewählt, dass der Kollektorstrom $I_C$ auf $75\,\%$ des maximal zulässigen Kollektorstroms $I_{C,\mathrm{max}}$ gesetzt wird. 
% Dies geschieht, indem der Widerstandswert $R_5$ so gewählt wird, dass der Kollektorstrom $I_C$ 15\,mA beträgt.
% Der Widerstandswert $R_5$ wird mit folgender Formel berechnet:
% \begin{equation}
%     R_5 = \frac{U_{CC} - U_{CE}}{I_C} = \frac{4.8\,\mathrm{V}}{15\,\mathrm{mA}} = 320\,\Omega
% \end{equation}
%
% Bei der Verwendung der E12-Reihe wird eine Faustregel angewendet, nach der man die Widerstandswerte auf die nächstgelegene E12-Reihe aufrundet. 
% Somit beträgt der errechnete Widerstandswert von $R_5$ 330\,\(\Omega\). Dadurch wird der in den Kollektor eingehende Strom begrenzt, was eine Überlastung des Transistors im Dauerbetrieb verhindert. 
% Bei der Wahl eines niedrigeren Widerstandswertes würde der Kollektorstrom $I_C$ den maximalen Kollektorstrom $I_{C,\mathrm{max}}$ überschreiten, was zu einer Überlastung des Transistors führen würde.

Das errechnete Ergebnis aus dem Kapitel \ref{subsec:rechnungkollektor} lässt sich anhand der Simulation verifizieren. 

Zuerst wird ein Sweep des Kollektorwiderstandes $R_5$ im Bereich von 100\,\(\Omega\) bis 4,7\,k\(\Omega\) durchgeführt.
Es ergibt sich folgender Verlauf des Kollektorstroms $I_C$ in Abhängigkeit des Kollektorwiderstandes $R_5$:
\begin{figure}[H]
    \centering
    \includegraphics[width=0.7\textwidth]{Pictures/RCollector.png}
    \caption{Verlauf des Kollektorstroms $I_C$ in Abhängigkeit des Kollektorwiderstandes $R_5$.}
    \label{fig:RCollector}
\end{figure}

Aus Abbildung \ref{fig:RCollector} ist zu erkennen, dass der Kollektorstrom $I_C$ mit steigendem Kollektorwiderstand $R_5$ abnimmt. Mit dem Marker m1 wird außerdem
gezeigt, dass der Kollektorstrom $I_C$ bei einem Widerstandswert von 317{,}416\,\(\Omega\) einen Wert von 15\,mA erreicht. Es wird der nächstgrößere Widerstandswert der E12-Reihe gewählt, also 330\(\Omega\), damit $I_C$ nicht überschritten wird.\,.
Die Ergebnisse der Rechnung werden somit bestätigt.

\subsection{Dimensionierung des Basis-Spannungsteilers}


Um einen Arbeitspunkt mit $I_C \approx 10\,\mathrm{mA}$ zu erreichen, wird der Basis-Spannungsteiler dimensioniert. Die Basis-Emitter-Spannung beträgt $U_{BE} \approx 0{,}7\,\mathrm{V}$.

% FÜR LUKAS
% Der Basisstrom berechnet sich zu:
% \begin{equation}
%     I_B = \frac{I_C}{\beta}
% \end{equation}
% Für einen typischen Verstärkungsfaktor $\beta = 100$ ergibt sich:
% \begin{equation}
%     I_B = \frac{10\,\mathrm{mA}}{100} = 0{,}1\,\mathrm{mA} = 100\,\mu\mathrm{A}
% \end{equation}
%
% Der Querstrom des Spannungsteilers $I_Q$ sollte mindestens das 10-fache des Basisstroms betragen:
% \begin{equation}
%     I_Q = 10 \cdot I_B = 1{,}0\,\mathrm{mA}
% \end{equation}
%
% Die Basisspannung $U_B$ ergibt sich zu:
% \begin{equation}
%     U_B = U_{BE} + U_E \approx 0{,}77\,\mathrm{V}
% \end{equation}
%
% Angenommen, die Betriebsspannung ist $U_{CC} = 4{,}8\,\mathrm{V}$, ergeben sich für die Widerstände $R_4$ (oben) und $R_3$ (unten):
% \begin{align}
%     R_3 &= \frac{U_B}{I_Q} = \frac{0{,}77\,\mathrm{V}}{1{,}0\,\mathrm{mA}} = 0{,}77\,\mathrm{k}\Omega \\
%     R_4 &= \frac{U_{CC} - U_B}{I_Q} = \frac{4{,}8\,\mathrm{V} - 0{,}77\,\mathrm{V}}{1{,}0\,\mathrm{mA}} = 4{,}03\,\mathrm{k}\Omega
% \end{align}
%
% Nach der Berechnung ergibt sich für $R_3$ ein Wert von $0{,}77\,\mathrm{k}\Omega$. In der praktischen Simulation mit ADS zeigte sich jedoch, dass mit diesem Wert ein negatives Gain bei der S-Parameter-Simulation auftritt. Daher wird $R_3$ nach der E12-Reihe auf $1{,}0\,\mathrm{k}\Omega$ erhöht, um einen stabilen Arbeitspunkt und ein positives Verstärkungsverhalten zu gewährleisten.
%
% Mit der E12-Reihe werden gewählt:
% \begin{align*}
%     R_3 &= 1{,}0\,\mathrm{k}\Omega \\
%     R_4 &= 4{,}7\,\mathrm{k}\Omega
% \end{align*}
%
% Damit ist der Spannungsteiler dimensioniert, sodass im Arbeitspunkt ein Kollektorstrom von ca. $10\,\mathrm{mA}$ zu erwarten ist und die Simulation ein positives Gain liefert.

Auch hier kann die im Kapitel \ref{subsec:basis-spannungsteiler} berechnete Dimensionierung überprüft werden. Beim Sweepen des Widerstandes $R_3$ im Bereich von 100\,\(\Omega\) bis 4,7\,k\(\Omega\) und des damit verbundenen Spannungsteilers, also auch durch das Sweepen des Widerstandes $R_4$, ergibt sich folgender Verlauf des Basisstroms $I_B$ in Abhängigkeit des Widerstandes $R_3$:
\begin{figure}[H]
    \centering
    \includegraphics[width=0.7\textwidth]{Pictures/RBase.png}
    \caption{Verlauf des Basisstroms $I_B$ in Abhängigkeit des Basiswiderstandes $R_3$.}
    \label{fig:RBase}
\end{figure}
Abbildung \ref{fig:RBase} bestätigt, dass der Basiswiderstand $R_3$ mit einem Wert von 1{,}0\,k\(\Omega\) einen Kollektorstrom $I_C$ von 10\,mA ergibt. Hierbei ist auch zu erkennen, dass sich ein $R_B$-Ratio von 4,243 ergibt. 
Das bedeutet, dass der Widerstand $R_4$ einen Widerstandswert von 4{,}243\,k\(\Omega\) haben sollte. Bei der Anpassung an die E12-Reihe wird der Widerstandswert auf 4{,}7\,k\(\Omega\) gewählt. 
Dies erfüllt die errechneten Bedingungen für die Widerstandswerte und bestätigt die ausgewählte Dimensionierung.

\section{S-Parameter-Simulation}
\label{sec:s_parameter_simulation}
Schließlich wird die S-Parameter-Simulation durchgeführt. Diese ist wichtig, um einen angemessenen Gain bei der Übertragung zu gewährleisten.
Die S-Parameter-Simulation wird in \ac{ADS} nach folgenden Schritten durchgeführt:
\begin{enumerate}
    \item Im ADS-Model wird die Kommentierung des S-Parameter-Controllers aufgehoben, also die S-Parameter-Einstellung wird aktiviert.
    \item Da jetzt die DC-Simulation überflüssig ist, wird diese deaktiviert.
    \item Vor der Simulation wird die Schrittweite des S-Parameter-Controllers auf 250 MHz gesetzt, sodass die Qualität des Frequenz-Sweeps verbessert wird.
    \includegraphics[height=0.7cm]{Pictures/simulate.png}
    \item Nach der Simulation wird das Ergebnis in einem neuen Fenster angezeigt. Es ergibt sich folgender Verlauf des Gains für verschiedene Frequenzen:
    
    \begin{figure}[h]
            \centering
            \includegraphics[width=0.7\textwidth]{Pictures/SParameter.png}
            \caption{Ergebnisse der S-Parameter-Simulation.}
            \label{fig:SParameter}
        \end{figure}
    
    Hier wählt man die Schaltfläche \enquote{Insert A New Marker} aus \includegraphics[height=0.5cm]{Pictures/marker.png}.
    Wir setzen den Marker auf den Punkt unserer Arbeitsfrequenz, also bei 1{,}25 GHz (siehe Abbildung \ref{fig:SParameter}). Der Marker m1 zeigt eine \textbf{Verstärkung (Gain) von 11,266 dB} an.
\end{enumerate}

Die Ergebnisse der Simulation werden bei der technischen Umsetzung der Schaltung überprüft. 
\subsection{Input- und Output Anpassung}
Wir beurteilen die Input- und Output-Anpassung der Schaltung mit Hilfe der durch die in der ADS-Simulation erstellte Grafik 
zur Auswertung der S-Parameter, hier dargestellt in Abbildung 3.4. Bei der Grafik handelt es sich um ein, bereits durch die 
Lehrveranstaltung Grundlagen der Nachrichtenübertragung bekanntes Diagramm,  das Smith Diagramm. Hier werden zwei 
Kurvenverläufe über eine Frequenz von 1 MHz bis hin zu 10 GHz dargestellt. Die rote Kurve S(1,1) zeigt den 
Eingangsreflexionsfaktor und die blaue Kurve S(2,2) den Ausgansgsreflexionsfaktor. 
Wünschenswert wäre ein Verlauf nahe des Zentrums bei $Z=1+j0$ da hier die Optimale Leistungsanpassung vorliegt. 

Damit die Kurven durch das Zentrum verläuft muss das Eingangs- und Ausgangsnetzwerk 
angepasst werden. Beide Netzwerke müssen auf den innenwiderstand von $Z_0=50\Omega$ angepasst werden.

\clearpage
% Erklärung zu [h] und [H]:
%
% [h] ist ein "placement specifier" für Gleitobjekte (z.B. figure, table) und bedeutet "here" – also möglichst an der aktuellen Stelle im Text.
% LaTeX kann das Objekt aber trotzdem verschieben, wenn es an dieser Stelle nicht passt.
%
% [H] (nur mit \usepackage{float}) erzwingt die Platzierung GENAU an der Stelle im Quelltext, ohne Ausweichmöglichkeiten.

% --- Abschnitt Theoretische Grundlagen (Platzhalter, falls nicht vorhanden) ---

% Hier können die theoretischen Grundlagen eingefügt werden.
