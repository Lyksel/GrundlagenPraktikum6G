
\section{Platinen Aufbau}
Die Platine ist mit mehreren Bauteilen ausgestattet, die bis auf drei selbstdimensionierten Widerständen
bereits vollständig bestückt ist. \\
% Auf dem Schaltplan  (R Widerstand, C Kondensator, J Stecker/Relais)
Die wichtigsten Bauteile sind:
%Frage: Generell wichtig oder nur für das was wir bei diesem Versuch machen? USB Buchse hier irrelevant....
\begin{itemize}
    \item J1: \textbf{[Platzhalter für Bauteilbeschreibung]}
    \item J40: \textbf{[Platzhalter für Bauteilbeschreibung]}
    \item Bauteil 3: \textbf{[Platzhalter für Bauteilbeschreibung]}
    \item Bauteil 4: \textbf{[Platzhalter für Bauteilbeschreibung]}
    \item Bauteil 5: \textbf{[Platzhalter für Bauteilbeschreibung]}
\end{itemize}



\section{Bestückung PCB}
Die in Kapitel 3 bestimmten Widerstände werden nun im Rahmen der praktischen Umsetzung der Schaltung auf die 
bereits vorbereitete Platine angebracht. Auf dem Bestückungsplan entspricht hier R47 R3 mit 1000 Ohm, R48 R4 mit 
4700 Ohm und R49 R5 mit 330 Ohm. Bei dem Löten der drei Widerständen wird auf eine saubere und präzise Löttechnik
geachtet um die gewollte elektrische, mechanische und HF-technische Funktion der Schaltung zu garantieren. 
\section{DC-Pegel Verifizieren}
\section{Kalibrierung}
%(erklärung warums tut)
\section{Vergleich zur Simulation}
\clearpage
