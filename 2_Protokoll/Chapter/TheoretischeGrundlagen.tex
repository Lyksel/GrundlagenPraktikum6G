

\section{Funktion eines HF-Verstärkers}
Ein Verstärker ist ein elektronisches Gerät, mit mindestens einem 
aktiven Bauelement wie zum Beispiel einem Transistor.
Das Ziel eines Verstärkers ist dass,das Ausgangssignal 
größer als das Eingangssignal ist. Da hierbei dem Signal leistung hinzugefügt
wird muss ein Verstärker eine eigene Energie Quelle haben.
\\
Besonders in der Hochfrequenztechnik (HF) spielt der Verstärker eine wichtige
Rolle. Soll zum Beispiel mithilfe einer Antenna noch in weiter entfernung ein Signal
gemessen werden muss dies zuerst verstärkt werden.
\\
Normalerweise werdem im Hochfrequenzbereich Frequenzen von 10 kHZ bis 100.000ß MHz
verstärkt.
\section{Arbeitspunkeinstellung}
\section{Bedeutung der S-Parameter}
\section{(rolle kopplungskodensator)}
blabla 
\clearpage