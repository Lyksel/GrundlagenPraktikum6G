\section{Ziel des Versuchs}
Im dritten Versuch im Rahmen des 6G-Hardwarelabors soll ein Coupled-Line-Filter entworfen und simuliert werden. Dazu wird das Filter zunächst in ADS entworfen, simuliert und optimiert. 
Anschließend soll das Filter in einem Messaufbau realisiert und die S-Parameter gemessen werden. Ziel ist es, die Eigenschaften des Filters zu verstehen und die Ergebnisse der Simulation mit den Messergebnissen zu vergleichen.

\section{Bedeutung von Coupled-Line-Filtern in 6G-Systemen}
Da 6G-Systeme bei hohen Frequenzen betrieben werden und zudem kompakte Bauformen erfordern, sind Coupled-Line-Filter eine wichtige Komponente. 6G erfordert massive MIMO-Technologien, die eine hohe Anzahl von Antennen und damit auch eine Vielzahl von Filtern benötigen.
Somit sind der Platzbedarf und die Effizienz der Filter von großer Bedeutung.
Sie ermöglichen die Realisierung von Filtern mit hoher Selektivität und geringer Einfügedämpfung, was für die Signalqualität in 6G-Systemen entscheidend ist.
Durch die Verwendung von Microstrip-Technologie können diese Filter auf kleinen Leiterplatten und somit in integrierten Schaltungen (MMICs) realisiert werden, was sie ideal für moderne Kommunikationssysteme macht. Es kommt außerdem zu einer geringen Dispersion der Phasengeschwindigkeit, was zu einer hohen Bandbreite und geringen Verzerrungen führt. Dies ist besonders wichtig für die Übertragung von hochfrequenten Signalen in 6G-Systemen, die eine hohe Datenrate und geringe Latenz erfordern.\footnote{Vgl. J.-S. Hong, M. J. Lancaster: \textit{Microstrip Filters for RF/Microwave Application}, siehe Literaturverzeichnis.}

Zunächst wird auf die theoretischen Grundlagen des Coupled-Line-Filters eingegangen.


\clearpage