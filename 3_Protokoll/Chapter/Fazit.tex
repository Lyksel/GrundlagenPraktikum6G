Im Verlauf dieses Versuchs konnten wir die grundlegende Funktionsweise eines Coupled-Line-Filters praktisch nachvollziehen und ebenfalls unser theoretisches Wissen erweitern. Es ist sehr interessant zu sehen, wie sich die anfangs unscheinbare Platine zu einem komplexen Konstrukt entwickelt, dessen einzelne Komponenten und deren Zusammenspiel wir nach und nach immer besser verstehen. Ebenfalls ist die Arbeit mit der Simulationssoftware ADS äußerst erkenntnisreich und hilft uns, die Fehler im praktischen Teil des Versuchs zu identifizieren und zu sehen, wie es unter optimalen Bedingungen aussehen müsste. Es gibt noch vieles zu lernen im Verlauf dieses Praktikums und wir sind gespannt auf das, was noch kommt.
\clearpage