
%----------------------------------------------------------------------------------------
    \section{Ziel des Versuchs}
    Bei dem Versuch 4 des Fachpraktikums 6G Hardwarelabor geht es um die Grundlagen der Modulation von Daten. 
    Zuerst wird eine Einführung in die Modulation gegeben, gefolgt von der Erklärung der verschiedenen Arten der Modulation. Nach einigen theoretischen Grundlagen zur Modulation werden diese in der Praxis umgesetzt.
    \section{Einführung in die Modulation von Daten}
    Vorweg sollte jedoch die Definition der Modulation erklärt werden, da sie für das Verständnis des Versuchs von Bedeutung ist.
    Die Modulation in der Kommunikationstechnik ist ein Verfahren, bei dem eine Eigenschaft einer Trägerfrequenz systematisch variiert wird, um Informationen (Nutzsignal) zu übertragen.
    Dabei wird das Nutzsignal auf die Trägerfrequenz aufmoduliert, um es über größere Entfernungen zu übertragen.
    Auf die Arten der Modulation wird es im Verlauf des Versuchs genauer eingegangen.
\clearpage