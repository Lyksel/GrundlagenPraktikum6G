\section{Zusammenfassung der wichtigsten Erkenntnisse}
Im Rahmen des Versuchs zur digitalen Amplitudenmodulation (On--Off~Keying) haben wir uns näher mit dem Thema auseinandergesetzt und wichtige Erkenntnisse gewonnen. Die Trägerfrequenz bleibt bei den zur Verfügung stehenden Platinen konstant bei etwa 1{,}200~GHz, während ihre Amplitude entsprechend des aktuell vorliegenden Pegels variiert wird.
Die Spektren bei verschiedenen Baudraten bestätigten das typische sinc-förmige Hauptkeulenprofil, dessen Breite in guter Übereinstimmung mit dem theoretischen Wert von $2 \times$~Baudrate liegt. 
Darüber hinaus konnte beobachtet werden, dass mit zunehmender Datenrate die erforderliche Bandbreite linear steigt.
Der eingesetzte Demodulator aus einer Gleichrichtung, einem Tiefpassfilter mit einer Eckfrequenz von 10~MHz sowie einer folgenden Verstärkungsstufe und einem Komparator lieferte ein weitestgehend störungsfreies Binärsignal; die Wahl einer Komparatorschwelle von 0,79~V ermöglichte dabei eine zuverlässige Unterscheidung zwischen High- und Low-Pegel. Kleinere Abweichungen zwischen den gemessenen und idealisierten Spektren sind durch praktische Versuchsbedingungen wie die begrenzte Auflösung des Spektrumanalysators, Ungenauigkeiten im lokalen Oszillator und Umgebungsrauschen zu erklären.

\section{Reflexion und mögliche Verbesserungen}
Bei der Überprüfung des Versuchsaufbaus zeigte sich insbesondere, dass eine präzisere Kalibrierung des Spektrumanalysators, beispielsweise mittels SOLT-Verfahren, systematische Messfehler deutlich reduzieren könnte.  s 
\clearpage