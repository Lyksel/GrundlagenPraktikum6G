\section{Modulationsarten}
Es gibt Analoge und Digitale Modulationsarten.
Um das Originalsignal zu modulieren wird meistens ein zusätliches Trägersignal benötigt.
Für Analoge Signale werden folgende Verfahren verwendet.

\subsection{Amplitudenmodulation AM}
Die Amplitude der Trägerschwingung wird durch das analoge Signale
$x(t)$ folgendermaßen verändert.
\begin{equation}
    a(t)=A_c(1+\mu x(t))
\end{equation}
Das AM-Signal wird beschrieben durch.
\begin{equation}
    x_c(t)=A_c(1+\mu x(t))cos(2\pi f_c t)
\end{equation}
\begin{align}
    A_c &\text{: Trägeramplitude} \\
    f_c &\text{: Trägerfrequenz} \\
    \mu &\text{: Modulationsindex, } 0 < \mu < 1
\end{align}


\subsection{Frequenzmodulation FM}

\section{Blockdiagramm einer Sendestrecke}
Im folgenden Abschnitt wird die Hochfrequenz-Übertragungsstrecke eines typischen Funksystems beschrieben. Bei der Grafik .. , handelt es sich um ein Blockdiagramm. Es zielt darauf ein grundlegendes
systematisches Verständinis aufzubauen um das gelernte auf unsere spezifische Hardware anwenden zu können. Die einzelnen Komponenten der Hochfrequenz-Übertragungsstrecke 
und deren Zusammenspiel wird im Anschluss näher erläutert und auf ihre Realiserung in unserer Hardware eingegangen.\\
\\Bild\\

\subsection{DAC}
Ein Digital-Analog-Wandler (eng. digital-to-analog converter, DAC) wandelt digitale Signale oder einzelne Werte in Analoge
Signale um. Bei einem digital Signal handelt es sich um ein zeit- und wertdiskretes Signal. Durch die Wandlung
in ein analoges Signal wird das Signal zeit- und wertkontinuierlich.  Dafür werden die Rechtecksignale des digitalen Eingangssignals mit Hilfe einer Fouriertransformation
in eine  kontinuirlich veränderliche Spannung transfomiert. Diese Wandlung ist erforderlich um das Signal über eine
Antenne aussenden zu können, da Antennen nur elektormagnetische Wellen abstrahlen können. \\

\subsection{LO und Mischer}
Der lokale Oszillaotr(eng. local oscillator, LO) erzeugt eine ungedämpfte hochfrequente Trägerschwingung. Diese Trägerschwingung 
wird benötigt, um das analoge Signal auf die gewünschte Frequenz zu bringen. Der LO kann in verschiedenen Frequenzen arbeiten,
abhängig von der Anwendung und dem gewünschten Frequenzbereich des Signals. Der Mischer übernimmt die Modulation des
Bandsignals auf eine Hochfrequenz. Dies geschieht durch die Multiplikation des Bandsignals mit der Trägerschwinung des LO.

\begin{equation}
    S_{xx}(t) \cdot S_{xx}(t) \rightarrow S_{xx}(t)
\end{equation}


\subsection{PA}
Der Leistungsverstärker (eng. power amplifier, PA) verstärkt das modulierte Signal auf eine Leistung, die für die Übertragung über eine Antenne 
geeignet ist. Die hohe Leistung ist notwedig um über eine größere Diszanz senden zu können und um Zuverlässigkeit und
Signalqualität zu gewährleisten. 

\subsection{Drahtlose Übertragung mit Antennen}
Die Sendeantenne strahlt das modulierte HF Signal Sxx als elektromagnetische Wellen in den Raum ab. DIese abgstrahlte Welle
breitet sich mit Lichtgeschwindigkeit aus und kann von Empfängerantennen empfangen werden. Die Empfängerantenne wandelt
die elektromagnetische welle wieder in eine elektrische Spannung um, die dann weiterverarbeitet werden kann. Diese entspricht
jedoch nicht mehr dem ursprünglichen Bandsignal, da es durch die Übertragungseinflüsse wie Dämpfung, Rauschen und Interferenzen
und vielen weiteren Einflüssen gedämpft und gestört wurde.

\subsection{LNA}
Bei dem LNA (eng. low noise amplifier) handelt es sich um einen rauscharmen HF-Verstärker. Das empfangene Signal Sxx ist
durch die bereits erwähnten Einflüsse sehr schwach und muss  zuerst verstärkt werden, um weiterverarbeitet werden. Daher
ist eine Verstärkung des Signals unmittlbar nach der Antennen zwingend notwendig. Der Vorteil des LNA gegenüber zu
anderen Verstärkern ist, dass er kein nennenswertes Rauschen hiinzufügt. Dies ist wichtig, da jedes zusätzliche Rauschen
die folgende Demolution erheblich erschweren würde. Ebenfall ist durch die Postion des LNA das empfangene Signal noch 
nicht durch andere elektrischen Komponenten verfälscht worden, was durch eine spätere Verstärkung zu rekonstruktionsproblemen
des eigentlichen Signals führen könnte.
\subsection{Demodulation}
Iwas mit Nyquisr Theorem und Abtastrate 

\subsection{ADC}
Nun muss das demodulisierte Signal wieder in ein digitales Signal umgewandelt werden, damit es weiterverarbeitet werden kann.
In unsere Schaltung wird dafür ... verwendet. 


\section{Mathematische Grundlagen: Fourier-Transformation}
\subsection{Betrag und zeitlicher Verlauf von Rechteckfunktion}
\subsection{Betrag und zeitlicher Verlauf von Sinusfunktion}
\subsection{Multiplikation der beiden Funktionen im Zeitbereich}


\section{Zusammenhang von Datenrate und Bandbreite}
blabla
\clearpage