\section{Modulationsarten}
Es gibt Analoge und Digitale Modulationsarten.
Um das Originalsignal zu modulieren wird meistens ein zusätliches Trägersignal benötigt.
Für Analoge Signale werden folgende Verfahren verwendet.

\subsection{Amplitudenmodulation AM}
Die Amplitude der Trägerschwingung wird durch das analoge Signale
$x(t)$ folgendermaßen verändert.
\begin{equation}
    a(t)=A_c(1+\mu x(t))
\end{equation}
Das AM-Signal wird beschrieben durch.
\begin{equation}
    x_c(t)=A_c(1+\mu x(t))cos(2\pi f_c t)
\end{equation}
\begin{align}
    A_c &\text{: Trägeramplitude} \\
    f_c &\text{: Trägerfrequenz} \\
    \mu &\text{: Modulationsindex, } 0 < \mu < 1
\end{align}


\subsection{Frequenzmodulation FM}

\section{Blockdiagramm einer Sendestrecke}
\subsection{DAC}
\subsection{LO}
\subsection{Mischer} 
\subsection{PA}
\subsection{Antennen}
\subsection{LNA}
\subsection{Demodulation}
\subsection{ADC}


\section{Mathematische Grundlagen: Fourier-Transformation}
\subsection{Betrag und zeitlicher Verlauf von Rechteckfunktion}
\subsection{Betrag und zeitlicher Verlauf von Sinusfunktion}
\subsection{Multiplikation der beiden Funktionen im Zeitbereich}


\section{Zusammenhang von Datenrate und Bandbreite}
blabla
\clearpage