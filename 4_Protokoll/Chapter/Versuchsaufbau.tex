\section{Verwendete Geräte und Materialien}
Im Versuch werden folgende Geräte verwendet:
\begin{itemize}
    \item \textbf{Keysight FieldFox Network Analyzer N9918A}: Zur Messung der S-Parameter der Sendeplatine.
    \item \textbf{Sendeplatine}: Die Sendeplatine wird verwendet, um die Trägerfrequenz zu erzeugen und modulierte Signale zu generieren.
    \item \textbf{Empfängerplatine}: Die Empfängerplatine empfängt die modulierten Signale und demoduliert sie, um die ursprünglichen Daten wiederherzustellen. Diese wird an den Fieldfox Network Analyzer N9918A angeschlossen, um die S-Parameter zu messen.
    \item \textbf{Rechner mit der Anwendung "HTerm (HyperTerminal)"}: Zur Steuerung der Sendeplatine und ggfs. zum Empfang der Daten von der Empfängerplatine. 
\end{itemize}
\clearpage
\section{Versuchsaufbau}
Zur Vertiefung des Verstädnisses des Versuchsaufbau wurde eine nichtqualitative Illustration der Versuchsanordnung erstellt (siehe Abbildung~\ref{fig:Versuchsanordnung}). Diese Abbildung zeigt die Anordnung der Geräte und die Verbindungen zwischen ihnen. Die Sendeplatine ist mit dem Fieldfox Network Analyzer verbunden, um die S-Parameter zu messen.
\begin{figure}[H]
    \centering
    \includegraphics[width=0.8\textwidth]{Pictures/Versuchsanordnung.jpg}
    \caption{Versuchsanordnung}
    \label{fig:Versuchsanordnung}
\end{figure}
\clearpage