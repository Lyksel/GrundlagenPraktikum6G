Jetzt wird zur praktischen Umsetzung der Signalverarbeitung im Empfänger übergegangen. Der Versuch besteht aus mehreren Teilen, die jeweils verschiedene Aspekte der Signalverarbeitung behandeln.

\section{Vermessung der Sensitivität des Empfängers} %Farhad
\subsection{Vermessung der Ausgangsleistung des Empfängers}
Eine grobe Veranschaulichung des Versuchsaufbaus für die Vermessung der Leistung ist in Abbildung \ref{fig:versuchsaufbau} zu sehen. 
Als erstes wird die Sensitivität des Empfängers vermessen. Dazu wird eine zweite Platine (Senderplatine) verwendet, die lediglich als ein Single-Tone-Sender fungiert. Die Ausgangsleistung des Senders wird mit dem Spektrumanalysator als Referenz gemessen. 
\begin{figure}[H]
    \centering
    \includegraphics[width=0.3\textwidth]{Pictures/VersuchsaufbauLeistung.jpg}
    \caption{Versuchsaufbau zur Vermessung der Ausgangsleistung des Empfängers}
    \label{fig:versuchsaufbau}
\end{figure}

Insgesamt wurde ein Leistungsspektrum des Senders gemessen, welches in Abbildung \ref{fig:sender} zu sehen ist. Zur genauen Vermessung sind die Werte bei den Harmonischen von 1,25 GHz mit Markern versehen, um die genaue Leistung zu bestimmen.

%\begin{figure}[H]
%    \centering
%    \includegraphics[width=0.8\textwidth]{Pictures/Sender.png}
%    \caption{Leistungsspektrum des Senders}
%    \label{fig:sender}
%\end{figure}

\subsection{Vermessung der Spannung am Kondensator C22}
Die DC-Spannung am Kondensator C22 wird jetzt auf der Empfängerplatine gemessen. Hierbei werden verschiedene Dämpfungsglieder verwendet, um die empfangene Leistung zu variieren. Eine grobe Veranschaulichung des Versuchsaufbaus dafür ist in Abbildung \ref{fig:versuchsaufbau2} zu sehen.
\begin{figure}[H]
    \centering
    \includegraphics[width=0.5\textwidth]{Pictures/VesuchsaufbauSpannung.jpg}
    \caption{Versuchsaufbau zur Vermessung der Spannung am Kondensator C22}
    \label{fig:versuchsaufbau2}
\end{figure}
Die Spannung am Kondensator C22 wird mit einem Multimeter gemessen. Die Ergebnisse sind in Tabelle \ref{tab:spannung} zu sehen.
\begin{table}[H]
    \centering
    \caption{Messwerte der Spannung am Kondensator C22}
    \begin{tabular}{|c|c|}
        \hline
        Dämpfungsglied & Spannung am C22 [mV] \\ \hline
        0 dB            & 386,47                 \\ \hline
        3 dB            & 383,20                 \\ \hline
        6 dB            & 379,85                 \\ \hline
        10 dB           & 375,10                 \\ \hline
        13 dB           & 372,90                 \\ \hline
        20 dB           & 368,90                 \\ \hline
        26 dB           & 368,27                 \\ \hline
    \end{tabular}
    \label{tab:spannung}
\end{table}

Wie man der Tabelle \ref{tab:spannung} entnehmen kann, fällt die Spannung am Kondensator C22 mit zunehmender Dämpfung des Senders ab. Dies ist zu erwarten, da die Leistung des Senders mit zunehmender Dämpfung verringert wird, was zu einer geringeren Spannung am Kondensator führt.

\subsection{Übertragungsfunktion des Empfängers (Ausgangsspannung/Eingangsleistung)}
Insgesamt stellt es sich heraus, dass diese Schaltung wie ein HF-Detektor (Schwellenwertdetektor) funktioniert. Das eingehende Hochfrequenzsignal wird gleichgerichtet und die DC-Spannung am Kondensator C22 ist proportional zur Leistung des eingehenden Signals,
also \( P \propto V_\text{in}^2 \), wobei \( V_\text{in} \) die Eingangsspannung ist. Die Beziehung zwischen der Spannung am Kondensator und der Leistung des eingehenden Signals ist quadratisch. Das liegt daran, dass das Eingangssignal des Empfängers hoch genug dafür ist, dass er insgesamt im linearen Bereich arbeitet. 

Hier sind jedoch einige Effekte zu beachten, die die Übertragungsfunktion beeinflussen können. Erstens fließt aufgrund der Arbeitspunkteinstellung des Transistors Q21 ein Strom durch den Widerstand R26. Die entsprechende Spannung beträgt ungefähr $368,27~mV$ und muss aus der Beziehung herausgerechnet werden, um die Übertragungsfunktion zu erhalten. Außerdem ist eine den Bauteilen der Schaltung charakteristische Konstante \( k \) zu berücksichtigen, die die Übertragungsfunktion beeinflusst. Diese Konstante kann experimentell bestimmt werden.

Die Übertragungsfunktion des Empfängers kann also wie folgt dargestellt werden:
\begin{equation}
    V_\text{out} = k \cdot \sqrt{P_\text{in}} - V_\text{offset}
\end{equation}
wobei \( V_\text{offset} \) die Spannung am Widerstand R26 ist, die den Arbeitspunkt des Transistors Q21 festlegt. 

\section{Operationsverstärkerschaltung} %Lukas
Der DC-Arbeitspunkt, wenn kein Signal anliegt, konnte messtechnisch auf etwa $358mV$ bestimmt werden.
\\
Nun wird mit einer zweiten Spannungsquelle vor der Diode eine Spannung eingespeist und dabei die Transferfunktion
$V_{out}/V_{in}$ vermessen. Der Spannungsbereich dieser Hilfsspannungsquelle $V_{in}$ wird von 0,9V in 0,01V-Schritten bis 1,4V erhöht.
Da das Protokoll so übersichtlich wie möglich gehalten werden soll, wird in der folgenden Tabelle nur ein kleiner Ausschnitt der
Messwerte gezeigt.
\begin{table}[h]
\centering
\begin{tabular}{|c|c|}
\hline
$V_{\text{in}}$ (V) & $V_{\text{out}}$ (mV) \\
\hline
0{,}9 & 0{,}15571 \\
1{,}0 & 0{,}24478 \\
1{,}1 & 0{,}33307 \\
1{,}2 & 0{,}42666 \\
1{,}3 & 0{,}52090 \\
1{,}4 & 0{,}61620 \\
\hline
\end{tabular}
\caption{Messwerte von $V_{\text{in}}$ und $V_{\text{out}}$ ohne Verstärkung}
\end{table}
\clearpage
Bei genauer Betrachtung der Messwerte wurde offensichtlich, dass die Platine beschädigt ist. Es findet keine Verstärkung
statt. Daher werden nun im weiteren Verlauf zwei Messungen betrachtet. Eine Messung ohne Verstärkung und eine Messung mit Verstärkung, die von 
unseren Kommilitonen durchgeführt wurde. Hier werden etwas mehr Messwerte zu einer groben Veranschaulichung benötigt. \\

\begin{table}[h]
\centering
\begin{tabular}{|c|c|}
\hline
$V_{\text{in}}$ (V) & $V_{\text{out}}$ (mV) \\
\hline
0{,}9 & 0{,}039 \\
1{,}0 & 0{,}047\\
1{,}03 & 0{,}077 \\
1{,}06 & 0{,}242 \\
1{,}1 & 0{,}531 \\
1{,}2 & 1{,}258 \\
1{,}3 & 1{,}994 \\
1{,}4 & 2{,}741 \\
\hline
\end{tabular}
\caption{Messwerte von $V_{\text{in}}$ und $V_{\text{out}}$ mit Verstärkung}
\end{table}
In Abbildung 4.3 ist die Transferfunktion $V_{out}/V_{in}$ dargestellt. Die geplotete Transferfunktion wurde
unter Berücksichtigung aller Messwerte erstellt.\\
(Eingehen auf 7,36 ungefähr 7,8V)





\begin{figure}[H]
    \centering
    \includegraphics[width=1\textwidth]{Pictures/Transferfunktion.jpf.png}
    \caption{Schaltplan des Empfängers}
    \label{fig:opamp_schaltung}
\end{figure}


\section{Komparatorschaltung} %Erik
Zuletzt wird überprüft, ob der Komparator beim Anliegen des HF-Tones durchschaltet und somit die Übertragung der digitalen Daten funktioniert.
Unsere Messwerte müssen hier nicht betrachtet werden, da der Schaltschwellwert des Komparators von
$U_{ref}= 0,79V$ nicht überschritten wird und somit der Komparator nicht schaltet und es auch keinen Sinn macht.
Wir beziehen uns hier wieder auf eine Messreihe unserer Kommilitonen. In der folgenden Tabelle wird die gemessene Spannung 
aufgetragen, die vor dem Komparator ankommt und wie sich der Komparator verhält. Das bedeutet, welche Ausgangsspannung
der Komparator in Abhängigkeit von der Eingangsspannung hat (LOW und HIGH).

\begin{table}[h]
\centering
\begin{tabular}{cc}
\textbf{Eingangsspannung} & \textbf{Ausgangsspannung} \\
\hline
0{,}811 & 0{,}242 \\
0{,}812 & 0{,}243 \\
0{,}816 & 0{,}244 \\
0{,}820 & 0{,}244 \\
0{,}821 & 0{,}245 \\
0{,}822 & 3{,}298 \\
\end{tabular}
\caption{Messwerte für Eingangsspannung und Ausgangsspannung}
\end{table}

Anhand der Tabelle kann man erkennen, dass der Komparator für Eingangsspannungen unterhalb von 0{,}822\,V auf LOW bleibt. Sobald die Eingangsspannung von 0{,}822\,V überschritten wird, schaltet der Ausgang auf HIGH. Die gemessene Schaltschwelle liegt somit etwas höher als der theoretische Wert $U_{ref} = 0{,}798$\,V bzw. der im Schaltplan angegebene Wert von 0{,}79\,V. Dies kann auf verschiedene Faktoren wie Bauteiltoleranzen oder Messungenauigkeiten zurückzuführen sein.