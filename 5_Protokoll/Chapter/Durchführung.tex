Jetzt wird zur praktischen Umsetzung der Signalverarbeitung im Empfänger übergegangen. Der Versuch besteht aus mehreren Teilen, die jeweils verschiedene Aspekte der Signalverarbeitung behandeln.

\section{Vermessung der Sensitivität des Empfängers} %Farhad
\subsection{Vermessung der Ausgangsleistung des Empfängers}
Eine grobe Veranschaulichung des Versuchsaufbaus für die Vermessung der Leistung ist in Abbildung \ref{fig:versuchsaufbau} zu sehen. 
Als erstes wird die Sensitivität des Empfängers vermessen. Dazu wird eine zweite Platine (Senderplatine) verwendet, die lediglich als ein Single-Tone-Sender fungiert. Die Ausgangsleistung des Senders wird mit dem Spektrumanalysator als Referenz gemessen. 
\begin{figure}[H]
    \centering
    \includegraphics[width=0.3\textwidth]{Pictures/VersuchsaufbauLeistung.jpg}
    \caption{Versuchsaufbau zur Vermessung der Ausgangsleistung des Empfängers}
    \label{fig:versuchsaufbau}
\end{figure}

Insgesamt wurde ein Leistungsspektrum des Senders gemessen, welches in Abbildung \ref{fig:sender} zu sehen ist. Zur genauen Vermessung sind die Werte bei den Harmonischen von 1,25 GHz mit Markern versehen, um die genaue Leistung zu bestimmen.

\begin{figure}[H]
    \centering
    \includegraphics[width=0.8\textwidth]{Pictures/Sender.jpg}
    \caption{Leistungsspektrum des Senders}
    \label{fig:sender}
\end{figure}

\subsection{Vermessung der Spannung am Kondensator C22}
Die DC-Spannung am Kondensator C22 wird jetzt auf der Empfängerplatine gemessen. Hierbei werden verschiedene Dämpfungsglieder verwendet, um die empfangene Leistung zu variieren. Eine grobe Veranschaulichung des Versuchsaufbaus dafür ist in Abbildung \ref{fig:versuchsaufbau2} zu sehen.
\begin{figure}[H]
    \centering
    \includegraphics[width=0.5\textwidth]{Pictures/VesuchsaufbauSpannung.jpg}
    \caption{Versuchsaufbau zur Vermessung der Spannung am Kondensator C22}
    \label{fig:versuchsaufbau2}
\end{figure}
Um die Spannung am Kondensator C22 in Abhängigkeit der Eingangsleistung zu bestimmen, benötigen wir die Sendeleistung des Senders,
aus Abbildung \ref{fig:sender} ist die Leistung bei 1.25 GHz bei etwa $-6,7~dBm$.Nach Versuch 1 ist die Dämpfung des Koaxialkabels
bei etwas $0,45~dB$, daraus ergibt sich eine Sendeleistung von $-6,25,~dBm$.
Die Spannung am Kondensator C22 wird mit einem Multimeter gemessen. Die Ergebnisse in Abhängigkeit der Dämpfung und Empfangsleistung sind in Tabelle \ref{tab:spannung} zu sehen.
\begin{table}[H]
    \centering
    \begin{tabular}{|c|c|c|}
        \hline
        Dämpfungsglied  & Empfangene Leistung [dBm] & Spannung am C22 [mV] \\ \hline
        0 dB + Kabel    & -6,683 & 386,47                 \\ \hline
        3 dB + Kabel    & -9,683 & 383,20                 \\ \hline
        6 dB + Kabel    & -12,683 & 379,85                 \\ \hline
        10 dB + Kabel   & -16,683 & 375,10                 \\ \hline
        13 dB + Kabel   & -19,683 & 372,90                 \\ \hline
        20 dB + Kabel   & -26,683 & 368,90                 \\ \hline
        26 dB + Kabel   & -32,683 & 368,27                 \\ \hline
    \end{tabular}
    \caption{Messwerte der Spannung am Kondensator C22}
    \label{tab:spannung}
\end{table}

Wie man der Tabelle \ref{tab:spannung} entnehmen kann, fällt die Spannung am Kondensator C22 mit zunehmender Dämpfung des Senders ab. Dies ist zu erwarten, da die Leistung des Senders mit zunehmender Dämpfung verringert wird, was zu einer geringeren Spannung am Kondensator führt.

\subsection{Übertragungsfunktion des Empfängers (Ausgangsspannung/Eingangsleistung)}
Insgesamt stellt es sich heraus, dass diese Schaltung wie ein HF-Detektor (Schwellenwertdetektor) funktioniert. Das eingehende Hochfrequenzsignal wird gleichgerichtet und die DC-Spannung am Kondensator C22 ist proportional zur Leistung des eingehenden Signals,
also \( P \propto V_\text{in}^2 \), wobei \( V_\text{in} \) die Eingangsspannung ist. Die Beziehung zwischen der Spannung am Kondensator und der Leistung des eingehenden Signals ist quadratisch. Das liegt daran, dass das Eingangssignal des Empfängers hoch genug dafür ist, dass er insgesamt im linearen Bereich arbeitet. 

Hier sind jedoch einige Effekte zu beachten, die die Übertragungsfunktion beeinflussen können. Erstens fließt aufgrund der Arbeitspunkteinstellung des Transistors Q21 ein Strom durch den Widerstand R26. Die entsprechende Spannung \textbf{$V_\text{offset}$} beträgt ungefähr $368,27~mV$ und muss aus der Beziehung herausgerechnet werden, um die Übertragungsfunktion zu erhalten. Außerdem ist eine den Bauteilen der Schaltung charakteristische Konstante \( k \) zu berücksichtigen, die die Übertragungsfunktion beeinflusst. Diese Konstante kann experimentell bestimmt werden.

Die Übertragungsfunktion des Empfängers kann also wie folgt dargestellt werden:
\begin{equation}
    V_\text{out} = k \cdot \sqrt{P_\text{in}} - V_\text{offset}
    \label{eq:uebertragungsfunktion}
\end{equation}
wobei \( V_\text{offset} \) die Spannung am Widerstand R26 ist, die den Arbeitspunkt des Transistors Q21 festlegt. 

Hier möchten wir den Wert von $k$ für unterschiedliche Eingangsleistungen und dadurch auch Ausgangsspannungen ausrechnen. Dafür stellen wir die Gleichung \ref{eq:uebertragungsfunktion} nach $k$ um. Es ergibt sich die folgende Gleichung:
\begin{equation}
    k = \frac{V_\text{out} + V_\text{offset}}{\sqrt{P_\text{in}}}
    \label{eq:k}
\end{equation}



Setzt man die in der Tabelle \ref{tab:spannung} berechneten werte in die Gleichung \ref{eq:k} ein, so ergeben sich folgende Werte:

\begin{table}[H]
    \centering
    \begin{tabular}{|c|c|c|c|}
        \hline
        Dämpfungsglied  & Empfangene Leistung [mW] & Spannung am C22 [mV] & k [$\frac{V}{\sqrt{mW}}$] \\ \hline
        0 dB + Kabel   & 0,215 & 386,47 & 1,63\\ \hline
        3 dB + Kabel   & 0,107 & 383,20 & 2,30\\ \hline
        6 dB + Kabel   & 0,054 & 379,85 & 3,22\\ \hline
        10 dB + Kabel  & 0,021 & 375,10 & 5,13\\ \hline
        13 dB + Kabel  & 0,011 & 372,90 & 7,07\\ \hline
        20 dB + Kabel  & 0,0022 & 368,90 & 15,72\\ \hline
        26 dB + Kabel  & 0,0005 & 368,27 & 32,94\\ \hline
    \end{tabular}
    \caption{Messwerte der Spannung am Kondensator C22}
    \label{tab:spannung}
\end{table}

Rechnet man den das arithmetische Mittel von $k$ aus, ergibt sich der folgende Wert:
\begin{equation}
    k = 9,72 \frac{V}{\sqrt{mW}}
\end{equation}

Somit ergibt sich die Folgende Gleichung:
\begin{equation}
    V_\text{out} = 9,72 \frac{V}{\sqrt{mW}} \cdot \sqrt{P_\text{in}} - 368,27 mV
\end{equation}
\section{Operationsverstärkerschaltung} %Lukas
Der DC-Arbeitspunkt, wenn kein Signal anliegt, konnte messtechnisch auf etwa $365mV$ bestimmt werden.
\\
Nun wird mit einer zweiten Spannungsquelle vor der Diode eine Spannung eingespeist und dabei die Transferfunktion
$V_{out}/V_{in}$ vermessen. Der Spannungsbereich dieser Hilfsspannungsquelle $V_{in}$ wird von 0,9V in 0,01V-Schritten bis 1,4V erhöht.
Da das Protokoll so übersichtlich wie möglich gehalten werden soll, wird in der folgenden Tabelle nur ein kleiner Ausschnitt der
Messwerte gezeigt.
\begin{table}[h]
\centering
\begin{tabular}{|c|c|}
\hline
$V_{\text{in}}$ (V) & $V_{\text{out}}$ (V) \\
\hline
0{,}9 & 0{,}15571 \\
1{,}0 & 0{,}24478 \\
1{,}1 & 0{,}33307 \\
1{,}2 & 0{,}42666 \\
1{,}3 & 0{,}52090 \\
1{,}4 & 0{,}61620 \\
\hline
\end{tabular}
\caption{Messwerte von $V_{\text{in}}$ und $V_{\text{out}}$ ohne Verstärkung}
\end{table}
Bei genauer Betrachtung der Messwerte wurde offensichtlich, dass die Platine beschädigt ist. Es findet keine Verstärkung
statt. Daher werden nun im weiteren Verlauf zwei Messungen betrachtet. Eine Messung ohne Verstärkung und eine Messung mit Verstärkung, die von 
unseren Kommilitonen durchgeführt wurde. Hier werden etwas mehr Messwerte zu einer groben Veranschaulichung benötigt. \\


\begin{table}[h]
\centering
\begin{tabular}{|c|c|}
\hline
$V_{\text{in}}$ (V) & $V_{\text{out}}$ (V) \\
\hline
0{,}9 & 0{,}039 \\
1{,}0 & 0{,}047\\
1{,}03 & 0{,}077 \\
1{,}06 & 0{,}242 \\
1{,}1 & 0{,}531 \\
1{,}2 & 1{,}258 \\
1{,}3 & 1{,}994 \\
1{,}4 & 2{,}741 \\
\hline
\end{tabular}
\caption{Messwerte von $V_{\text{in}}$ und $V_{\text{out}}$ mit Verstärkung}
\end{table}
Neu: Diese Messwerte sind jedoch in dieser Form noch nicht weiter verwertbar für den weiteren Verlauf 
des Praktikums. Wenn wir die über die Diode Q21 abfallende Spannung $U_{\text{D}}$ von der gemessenen
Spannung abziehen, erhalten wir die Spannung die am Eingang unseres Operationsverstärkers anliegt.
Diese erhält man wenn man den DC-Arbeitspunkt, wenn kein Signal anliegt, in die Formel zur Ausgangsspannung
in Ábhängigkeit von der Eingangsspannung einsetzt.
Die Ausgangsspannung $U_\text{out}$ ohne Signal berechnet sich zu:
\[
U_\text{out,0} = 7{,}8 \cdot 0{,}365\,\text{V} - 2{,}04\,\text{V} = 0{,}807\]
Dies ist genauer als die Spannung $U_\text{D'}$ die 
Dies liegt nahe an der in der Aufgabenstellung angenommenen Dioden-Flussspannung $U_\text{D'} = 0,8V$,
\clearpage
In Abbildung 4.4 ist die Transferfunktion $V_{out}/V_{in}$ dargestellt. Die geplotete Transferfunktion wurde
unter Berücksichtigung aller Messwerte erstellt.\\

\begin{figure}[H]
    \centering
    \includegraphics[width=1\textwidth]{Pictures/Transferfunktion.jpf.png}
    \caption{Transferfunktion}
    \label{fig:opamp_schaltung}
\end{figure}
In der Praxis bekommen wir ein Übertragungsverhältnis von 7,36.
In der Theorie liegt das Verhältnis höher, etwa bei 7,8. 
Dies kann auf verschiedene Faktoren wie Bauteiltoleranzen, Messungenauigkeiten und Verluste zurückzuführen sein.
\\
Bei der kaputten Platine spiegelt sich die fehlende Verstärkung schön wider, da der Übertragungsfaktor etwa bei eins liegt,
leicht darunter aufgrund von Verlusten und den vorher genannten Faktoren.

\section{Komparatorschaltung} %Erik
Zuletzt wird überprüft, ob der Komparator beim Anliegen des HF-Tones durchschaltet und somit die Übertragung der digitalen Daten funktioniert.
Unsere Messwerte müssen hier nicht betrachtet werden, da der Schaltschwellwert des Komparators von
$U_{ref}= 0,79V$ nicht überschritten wird und somit der Komparator nicht schaltet. Deshalb hat es keinen weiteren Sinn, die Messung auszuwerten, da daraus keine Schlüsse gezogen werden können.
Wir beziehen uns anschließend wieder auf eine Messreihe unserer Kommilitonen. In der folgenden Tabelle wird die gemessene Spannung
aufgetragen, die vor dem Komparator ankommt und wie sich der Komparator verhält. Das bedeutet, welche Ausgangsspannung
der Komparator in Abhängigkeit von der Eingangsspannung hat (LOW und HIGH).

\begin{table}[h]
\centering
\begin{tabular}{cc}
\textbf{Eingangsspannung} & \textbf{Ausgangsspannung} \\
\hline
0{,}811 & 0{,}242 (LOW) \\
0{,}812 & 0{,}243 (LOW) \\
0{,}816 & 0{,}244 (LOW) \\
0{,}820 & 0{,}244 (LOW) \\
0{,}821 & 0{,}245 (LOW) \\
0{,}822 & 3{,}298 (HIGH) \\
\end{tabular}
\caption{Messwerte für Eingangsspannung und Ausgangsspannung}
\end{table}

Anhand der Tabelle kann man erkennen, dass der Komparator für Eingangsspannungen unterhalb von 0{,}822\,V auf LOW bleibt. Sobald die Eingangsspannung von 0{,}822\,V überschritten wird, schaltet der Ausgang auf HIGH. Die gemessene Schaltschwelle liegt somit etwas höher als der theoretische Wert $U_{ref} = 0{,}798$\,V bzw. der im Schaltplan angegebene Wert von 0{,}79\,V. Dies kann auf die zuvor genannten Faktoren zurückgeführt werden.