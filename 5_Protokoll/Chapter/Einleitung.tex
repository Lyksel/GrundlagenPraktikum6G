\section{Relevanz des Empfängers}
    Ein Signalempfänger ist ein Gerät, das in der heutigen Digital- und Kommunikationstechnik eine zentrale Rolle spielt. Er empfängt Signale, die über verschiedene Medien wie Luft, Kabel oder Glasfaser übertragen werden, und wandelt diese in nutzbare Informationen um.
    Seine Funktion ist es, die empfangenen Signale zu verstärken, zu filtern und zu demodulieren, um die ursprünglichen Daten wiederherzustellen. Dies ist besonders wichtig in der drahtlosen Kommunikation, wo Signale durch verschiedene Störungen und Rauschen beeinträchtigt werden können.
\section{Ziel des Versuchs}
    Das Ziel des Versuches 5 im 6G-Hardwarelabor ist es, die Grundlagen der Modulation und Demodulation von Daten zu verstehen und praktisch anzuwenden. Hierzu sollen die einzelnen Funktionsblöcke der Empfängerplatine untersucht werden, um deren Funktionalität zu verstehen. Daraufhin wird die Sensitivität der Empfängerplatine in Abhängigkeit von den Dämpfungsgliedern untersucht. Zuletzt wird die Funktion des Komparators getestet.
\clearpage