\section{Zusammenfassung der wichtigsten Erkenntnisse}
Im Rahmen des Versuchs zur Signalverarbeitung im Empfänger haben wir uns näher mit der Funktionsweise von Modulation und Demodulation beschäftigt. Durch eine nähere Analyse des Empfängers und seiner einzelnen Komponenten, unter Anderem des Downkonverters, des Operationsverstärkers und des Komparators
konnten wir die Grundlagen der Signalverarbeitung verstehen und deren praktische Anwendung nachvollziehen. Der Operationsverstärker spielt dabei durch eine geschickte Anordnung der Spannungsteiler an den Widerständen eine entscheidende Rolle bei der Verstärkung des Signals auf den gewünschten Bereich um die Schwellspannung des Komparators. Der Komparator wiederum wandelt das analoge Signal in ein digitales Signal um, welches von einem Computer weiterverarbeitet werden kann.
Der Einblick in die Funktionsweise des Empfängers und auch den Einfluss der Dämpfungsglieder auf die Signalverstärkung hat uns ein besseres Verständnis für die Signalverarbeitung in der Kommunikationstechnik vermittelt.
\section{Reflexion und mögliche Verbesserungen}
Eine Problemquelle bei dem Versuch 5 war für unsere Gruppe eine unzureichende Signalverstärkung, die dazu führte dass die Spannung am Kondensator C22 nicht den erwarteten Wert erreicht hat. Dies führte zu Schwierigkeiten bei der Funktion des Komparators, da die Schwellenspannung nicht erreicht wurde. Eine mögliche Verbesserung wäre eine Sicherstellung der richtigen Bestückung der Platine, um eine optimale Signalverstärkung zu gewährleisten.

\clearpage