\section{Zusammenfassung der wichtigsten Erkenntnisse}
Im Rahmen des Versuches konnten wir unser Verständnis über den Empfänger noch weiter vertiefen. Durch eine nähere Analyse des Empfängers und seiner einzelnen Komponenten, unter anderem des Downkonverters, des Operationsverstärkers und des Komparators,
konnten wir die Grundlagen der Signalverarbeitung verstehen und deren praktische Anwendung nachvollziehen. Der Operationsverstärker spielt dabei durch eine geschickte Anordnung der Spannungsteiler an den Widerständen eine entscheidende Rolle bei der Verstärkung des Signals auf den gewünschten Bereich um die Schwellspannung des Komparators zu überschreiten. Der Komparator wiederum wandelt das analoge Signal in ein digitales Signal um, welches von einem Computer weiterverarbeitet werden kann.
Der Einblick in die Funktionsweise des Empfängers und auch in den Einfluss der Dämpfungsglieder auf die Signalverstärkung hat uns ein besseres Verständnis für die Signalverarbeitung in der Kommunikationstechnik vermittelt.
\section{Reflexion und mögliche Verbesserungen}
Eine Problemquelle bei Versuch fünf war für unsere Gruppe eine defekte Platine. Dies führte zu einer nicht vorhandenen Verstärkung, sodass die Spannung am Kondensator C22 nicht den erwarteten Wert erreicht hat. Deshalb waren unsere Messreihen nicht aussagekräftig.
Somit konnten wir Aufgaben nicht lösen und mussten Messreihen von Kommilitonen zu Hilfe nehmen.
\clearpage