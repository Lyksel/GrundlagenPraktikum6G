\begin{thebibliography}{1}
\bibitem{microstrip_filters}
Hong, J.-S.; Lancaster, M. J.: \emph{Microstrip Filters for RF/Microwave Applications}. 2., überarb. Aufl. Hoboken, NJ: Wiley-Blackwell, 2011. ISBN 978-0-470-40877-3.

\bibitem{SchaltplanPCBV4}
Haussmann, Simon: \emph{Schaltplan\_PCB\_V4}, 15. April 2024. Institut für Robuste Leistungshalbleitersysteme, Universität Stuttgart. Online verfügbar unter: \url{https://ilias3.uni-stuttgart.de/ilias.php?baseClass=ilrepositorygui&cmdNode=z5:o1&cmdClass=ilObjFileGUI&cmd=sendfile&ref_id=4067155} (abgerufen am 19.05.2025).

\bibitem{NT-Skript}
Hesselbarth, Jan: Allgemein Wissen. In: Nachrichtentechnik 1. Nachrichtentechnik 1 Skript. Online verfügbar unter: \url{Nachrichtentechnik 1 Skript}, abgerufen am 20.05.2025.


\bibitem{Amplitudenmodulation}
Elektronik-Kompendium: \emph{Amplitudenmodulation (AM)}. Stand: 12. Dezember 2023.  
Online verfügbar unter: \url{https://www.elektronik-kompendium.de/sites/kom/0211195.htm} (abgerufen am 17.06.2025).

\bibitem{Modulationsverfahren}
Roppel, Thomas: \emph{Analoge Modulationsverfahren}. Hochschule Schmalkalden, Fakultät Elektrotechnik.  
Online verfügbar unter: \url{https://www.hs-schmalkalden.de/fileadmin/portal/Dokumente/Fakult%C3%A4t_ET/Personal/Roppel/Buch/Analoge_Modulationsverfahren.pdf} (abgerufen am 17.06.2025).

\bibitem{Amplitudenmodulation}
50Ohm.de: \emph{Amplitudenmodulation (AM)}. Ohne Datum.  
Online verfügbar unter: \url{https://50ohm.de/A_pm.html} (abgerufen am 20.06.2025).

\bibitem{Baudrate}
Wikipedia: \emph{Symbolrate}. Letzte Änderung am 6. Juni 2024.  
Online verfügbar unter: \url{https://de.wikipedia.org/wiki/Symbolrate} (abgerufen am 19.06.2025).

\bibitem{dewetron-abtastrate}
DEWETRON GmbH: \emph{Eine Einführung in Abtastrate, Bandbreite \& Co.}, 12. April 2022.  
Online verfügbar unter: \url{https://www.dewetron.com/de/news/eine-einfuehrung-in-abtastrate-bandbreite-co/} (abgerufen am 18.06.2025).

\bibitem{LNA}
Wikipedia: \emph{Low Noise Amplifier}. Letzte Änderung am 18. Mai 2024.  
Online verfügbar unter: \url{https://de.wikipedia.org/wiki/Low_Noise_Amplifier} (abgerufen am 17.06.2025).

\bibitem{DAC}
Wikipedia: \emph{Digital-Analog-Wandler}. Letzte Änderung am 8. Mai 2024.  
Online verfügbar unter: \url{https://de.wikipedia.org/wiki/Digital-Analog-Umsetzer} (abgerufen am 16.06.2025).

\bibitem{Verstärker}
Wikipedia: \emph{Verstärker (Elektrotechnik)}. Letzte Änderung am 30. April 2024.  
Online verfügbar unter: \url{https://de.wikipedia.org/wiki/Verst%C3%A4rker_(Elektrotechnik)} (abgerufen am 18.06.2025).

\bibitem{fourier-uni-muenster}
Krömer, Elisabeth; Schmitz, René: \emph{Fourier-Transformation}. Physikalisch-Chemisches Praktikum, Universität Münster.  
Online verfügbar unter: \url{https://www.uni-muenster.de/imperia/md/content/physikalische_chemie/praktikum/fourier__transformation__kr_mer__elisabeth__schmitz__rene_.pdf} (abgerufen am 20.06.2025).

\bibitem{Frequenzmodulation}
Elektronik-Kompendium: \emph{Frequenzmodulation}. Stand: 12. Dezember 2023.  
Online verfügbar unter: \url{https://www.elektronik-kompendium.de/sites/kom/0401251.htm} (abgerufen am 16.06.2025).

\bibitem{Phasenmodulation}
Elektronik-Kompendium: \emph{Phasenmodulation}. Stand: 12. Dezember 2023.  
Online verfügbar unter: \url{https://www.elektronik-kompendium.de/sites/kom/0402021.htm} (abgerufen am 17.06.2025).

\end{thebibliography}

%https://de.wikipedia.org/wiki/Friis-Formel
\clearpage
