\section{Verwendete Geräte}
Im Versuch werden folgende Geräte verwendet:
\begin{itemize}
    \item \textbf{Keysight FieldFox Network Analyzer N9918A}: Zur Messung der S-Parameter der Sendeplatine.
    \item \textbf{Sendeplatine}: Die Sendeplatine wird verwendet, um die Trägerfrequenz zu erzeugen und modulierte Signale zu generieren.
    \item \textbf{Empfängerplatine}: Die Empfängerplatine empfängt die modulierten Signale und demoduliert sie, um die ursprünglichen Daten wiederherzustellen. Sie wird an den FieldFox Network Analyzer N9918A angeschlossen, um die S-Parameter zu messen.
    \item \textbf{Multimeter VOLTCRAFT VC871}: Zur Messung der Spannung an den verschiedenen Punkten der Schaltung.
    \item \textbf{Dämpfungsglieder}: Zur Variation der Leistung des Senders.
    \item \textbf{Koaxialkabel}: Zur Verbindung der Sende- und Empfängerplatine mit dem FieldFox Network Analyzer.
\end{itemize}
\section{Messaufbau}
Wie man später in der Versuchsbeschreibung erfahren wird, besteht die Messung aus folgenden Schritten:
\begin{enumerate}
    \item Zuerst wird die Sendeplatine an die Versorgungsspannung angeschlossen, um den Sender zu aktivieren.
    \item Es wird eine SMA-Verbindung zwischen der Sendeplatine und dem FieldFox hergestellt, um die Sendeleistung zu messen.
    \item Nach einer Messung der Leistung des Senders wird auch die Empfängerplatine an eine Versorgungsspannung angeschlossen, die Sendeplatine vom FieldFox getrennt und mit der Empfängerplatine über SMA-Kabel verbunden. Der Sender wird mit einer Spannung von $4,8~V$ gespeist.
    \item Eines nach dem anderen werden die Dämpfungsglieder in den Signalweg eingefügt, um die Leistung des Senders zu variieren. Die Spannung am Kondensator C22 der Empfängerplatine wird hierbei mit einem Multimeter gemessen.
    \item Zuletzt wird eine zusätzliche Spannung am Einspeisepunkt des Komparators angelegt und zwischen $0,9~V$ und $1,4~V$ variiert, um die Funktion des Komparators zu testen. Die Spannung am Kondensator C22 wird hierbei ebenfalls gemessen.
\end{enumerate}
\clearpage