\section{Task 2}
Für diesen Versuch werden zwei identische Tranceiver Platinen verwendet, die bereits im Verlauf dieses Praktikums genauer betrachtet wurden.
Die beiden Platinen haben folgende Spezifikationen:


\begin{table}[h!]
    \centering
    \begin{tabular}{|c|c|c|c|}
        \hline
         & J1 & J2 & Funktion \\
        \hline
        Platine A &  &  & GHz \\
        Platine B & 100 & 100 & m \\
        \hline
    \end{tabular}
    \caption{Spezifikationen der beiden Platinen}
\end{table}

\subsection{Plantinen am Computer anschließen}
Nur wurde die Eine Platine am Computer angeschlossen und die andere Platine an den Laptop.
Mithilfe von Hterm werden Daten in Form von ASCII von der einen Platine an die andere gesendet.
Und mit Hterm wieder ausgelesen.
Wichtig in dieser Prozedur war das verbinden beider Jumper an der Sender Platine und das entfernen beider
Jumper an der Empfänger Platine. 
Wären die Jumper bei der Empfänger Platine verbunden gewesen, wäre das problem das die Platine auch 
ein Signal gesendet hätte und dieses auch wieder empfangen hätte