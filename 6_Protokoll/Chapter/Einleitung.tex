\section{Relevanz der Funkübertragung im Alltag}
Jeden Tag nutzen wir Funkübertragungen in verschiedenen Formen, sei es durch WLAN, Bluetooth oder Mobilfunk. Diese Technologien ermöglichen es uns, Daten über große Entfernungen zu übertragen, ohne physische Verbindungen herstellen zu müssen. Die Grundlagen der Modulation sind entscheidend für die Entwicklung und Verbesserung dieser Technologien.
Die \ac{6G} der Funkübertragung ist die neueste Generation der Funkkommunikation, die eine höhere Datenrate, geringere Latenz und verbesserte Zuverlässigkeit verspricht. Sie befindet sich aktuell in der Entwicklung. 
Jedoch besteht bei dem Frequenzbereich von \ac{6G}, beginnend mit Sub-6 GHz (unter 6 GHz) bis hin zum THz-Bereich (100 GHz - 1 THz), 
die Herausforderung, dass die Signale bei höheren Frequenzen stärker gedämpft werden und somit eine höhere Signalstärke erforderlich ist, um eine zuverlässige Kommunikation zu gewährleisten.
Deswegen ist es leichter, eine beispielhafte Funkübertragung bei kleineren Abständen zu dimensionieren, um das Wissen bei größeren Strecken anwenden zu können.
\section{Ziel des Versuchs}
Das Ziel des heutigen und des letzten Versuchs ist es, eine Bildübertragung über eine Funkverbindung zu realisieren und dabei die Grundlagen der Bildübertragung mithilfe von 6G zu verstehen.
Dies bietet einen guten Einblick in die digitale Kommunikation und die Modulation von Signalen, die für die Übertragung von Daten über Funkverbindungen unerlässlich sind.
\clearpage 