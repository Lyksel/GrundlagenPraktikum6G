Man könnte sagen, dieser Versuch war die Krönung unseres Praktikums. 
Auf einmal haben wir alle Komponenten, die wir über die letzte Zeit kennengelernt haben, in einem Versuch vereinigt.
Nun erkennt man, dass z.\,B. der LNA, der am Anfang so aufwendig berechnet und simuliert wurde, seine Daseinsberechtigung hat.
Ebenso alle anderen Komponenten, die in ihrer Gesamtheit unerlässlich für eine erfolgreiche Funkübertragung sind.
Es war auch interessant zu sehen, dass die am Anfang wenig beachteten Jumper in der Lage waren, die Funktionalität
unserer Platine zu ändern. Somit konnten wir mit zwei Transceivern ein Funksystem aufbauen, welches uns ermöglichte, eine Funkverbindung herzustellen,
mit der man Bilder übertragen konnte.
Zusammenfassend lässt sich sagen, dass das Praktikum trotz zahlreicher Probleme und viel Schweiß und Tränen uns ein weitaus besseres Verständnis für die Nachrichtentechnik vermittelt hat.
Uns wurden die Herausforderungen bewusst, die der fortwährenden Verbesserung der 6G-Technologie unterliegen, um dem Nutzer das bestmögliche Ergebnis zu ermöglichen.