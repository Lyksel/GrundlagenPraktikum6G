\section{Zusammenfassung der wichtigsten Erkenntnisse}
Im Rahmen des Versuchs zur End-to-End-Datenübertragung haben wir uns näher mit der digitalen Amplitudenmodulation (On--Off~Keying) auseinandergesetzt und wichtige Erkenntnisse gewonnen.
Außerdem konnten wir einen Einblick in die Übertragungsqualität und die Herausforderungen bei der Datenübertragung über Funkverbindungen gewinnen. Die Konvertierung der Bilder in ASCII-Codes und deren Übertragung über die Funkverbindung ermöglichten es uns, die Grundlagen der digitalen Kommunikation zu verstehen und anzuwenden.
Es war eine sehr lehrreiche und dabei auch praxisnahe Erfahrung, da wir die Schaltung tatsächlich angewendet haben.
\section{Reflexion und mögliche Verbesserungen}
Die Durchführung des Versuchs verlief insgesamt reibungslos, jedoch gab es einige Herausforderungen, die wir meistern mussten. So war es anfangs schwierig, die Funktion der Jumper zu verstehen und korrekt anzuwenden, um die Datenübertragung zwischen den Platinen zu ermöglichen.
Nichtsdestotrotz war es eine interessante Erfahrung, die uns ein besseres Verständnis für die digitale Kommunikation und die Modulation von Signalen im Frequenzbereich von 6G vermittelt hat.
\clearpage
