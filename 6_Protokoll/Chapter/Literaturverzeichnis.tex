\begin{thebibliography}{1}

\bibitem{SchaltplanPCBV4}
Haussmann, Simon: \emph{Schaltplan\_PCB\_V4}, 15. April 2024. Institut für Robuste Leistungshalbleitersysteme, Universität Stuttgart. Online verfügbar unter: \url{https://ilias3.uni-stuttgart.de/ilias.php?baseClass=ilrepositorygui&cmdNode=z5:o1&cmdClass=ilObjFileGUI&cmd=sendfile&ref_id=4067155} (abgerufen am 19.05.2025).

\bibitem{NT-Skript}
Hesselbarth, Jan: Allgemein Wissen. In: Nachrichtentechnik 1. Nachrichtentechnik 1 Skript. Online verfügbar unter: \url{Nachrichtentechnik 1 Skript}, abgerufen am 20.05.2025.

\bibitem{Friis-Formel}
Wikipedia: \emph{Friis-Formel}. Letzte Änderung am 3. März 2025.  
Online verfügbar unter: \url{https://de.wikipedia.org/wiki/Friis-Formel} (abgerufen am 30.06.2025).

\bibitem{UART-Verstehen}
Rohde \& Schwarz: \emph{UART verstehen}.  Online verfügbar unter: \url{https://www.rohde-schwarz.com/de/produkte/messtechnik/essentials-test-equipment/digital-oscilloscopes/uart-verstehen_254524.html} (abgerufen am 07.07.2025).

\bibitem{Bitfehlerquote}
Wikipedia: \emph{Bit error rate}. Letzte Änderung am 27. Juni 2025.  
Online verfügbar unter: \url{https://en.wikipedia.org/wiki/Bit_error_rate} (abgerufen am 08.07.2025).

\bibitem{Bitfehlerquote}
Wikipedia: \emph{MIMO (Nachrichtentechnik)}. Letzte Änderung am 16. Januar 2025.  
Online verfügbar unter: \url{https://de.wikipedia.org/wiki/MIMO_(Nachrichtentechnik)} (abgerufen am 08.07.2025).

\bibitem{Diversity scheme}
Wikipedia: \emph{Diversity scheme}. Letzte Änderung am 19. September 2024.
Online verfügbar unter: \url{https://en.wikipedia.org/wiki/Diversity_scheme} (abgerufen am 08.07.2025)

\end{thebibliography}

\clearpage
