\section{Analyse der Tranceiver-Platinen}
\subsection{Funktion der Jumper auf der Tranceiver-Platine}
\subsubsection{J1}
\subsubsection{J2}
\subsection{Serielles Protokoll UART}
Hier RF-Signalübertragung bei einer digitalen 1.
\subsection{BER}
Bit Error Rate (BER) oder in Deutsch Bitfehlerrate ist eine angabe die die
Anzahl der falsch empfangenen Bits ins Verhältnis zu den gesamt empfangenen Bits setzt.
\begin{equation}
    BER = \frac{N_{Anzahl fehlerhaft empfangener Bits}}{N_{Anzahl empfangener Bits}}
\end{equation}
\subsection{Beispiel}
Wir betrachten ein Beispiel, folgende Sequenz wird gesendet:
\begin{equation}
    10111001
\end{equation}
Empfangen wurde aber:
\begin{equation}
      %1\textcolor{red}{0}01\textcolor{red}{1}0\textcolor{red}{1}1
      232

\end{equation}

\section{Pegelplanrechnung}
\subsection{Rekapitulation des Link-Budgets}
\subsection{Rekapitulation der Sensitivität}
\subsection{Berechnung des Pegeplans}